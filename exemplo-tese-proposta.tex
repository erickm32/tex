\documentclass[tese-proposta,nocipinfo]{texufpel}
%nocipinfo para não aparecer os dados da CIP no Resumo

\usepackage[utf8]{inputenc} % acentuacao
\usepackage{graphicx} % para inserir figuras
\usepackage[T1]{fontenc}

\hypersetup{
    hidelinks, % Remove coloração e caixas
    unicode=true,   %Permite acentuação no bookmark
    linktoc=all %Habilita link no nome e página do sumário
}

\unidade{Centro de Desenvolvimento Tecnológico}
\programa{Programa de Pós-Graduação em Computação}
\curso{Ciência da Computação}

\title{Um Blabla Blablabla com Aplicações em Blablabla}

\author{UltimoNome}{Nome Sobrenome de}
\advisor[Prof.~Dr.]{Aguiar}{Marilton Sanchotene de}
%\coadvisor[Prof.~Dr.]{Aguiar}{Marilton Sanchotene de}
%\collaborator[Prof.~Dr.]{Aguiar}{Marilton Sanchotene de}

%Palavras-chave em PT_BR
\keyword{palavrachave-um}
\keyword{palavrachave-dois}
\keyword{palavrachave-tres}
\keyword{palavrachave-quatro}

%Palavras-chave em EN_US
\keywordeng{keyword-one}
\keywordeng{keyword-two}
\keywordeng{keyword-three}
\keywordeng{keyword-four}

\begin{document}

%\renewcommand{\advisorname}{Orientadora}           %descomente caso tenhas orientadora
%\renewcommand{\coadvisorname}{Coorientadora}      %descomente caso tenhas coorientadora

\maketitle 
\sloppy

%Resumo em Portugues (no maximo 1 pagina)
\begin{abstract}
  Apresentar aqui uma breve Introdução ao Problema que está se
  pretendendo resolver ou abordar. Além disso, nesta Seção,
  apresenta(m)-se o(s) principal(is) objetivo(s) do projeto e,
  portanto, a(s) principal(is) contribuição(ções).
\end{abstract}

\chapter{Motivação}
% (ENTRE 1 e 2 PÁGINAS)

Nesta Seção, apresenta-se um breve histórico da área de concentração
da Tese, partindo do tema mais abrangente até chegar
especificamente no assunto do Projeto. Além disso, apresenta-se a
justificativa para a realização do trabalho, sua importância acadêmica
ou para comunidade e grau de inovação. Poderá também apresentar as
distinções entre o trabalho atual e outros trabalhos já realizados.

\chapter{Objetivos e Resultados}
% (ENTRE 1 e 3 PÁGINAS)

Nesta Seção, apresentam-se o objetivo Geral e os objetivos Específicos
da Tese. Os objetivos não devem ser confundidos com as
atividades. Para a definição das atividades, deve-se partir dos
objetivos determinados nesta Seção. O objetivo Geral do Projeto
necessariamente deve ser algum resultado prático (implementação) ou
teórico (modelos formais ou especificações ou validações) produto da
pesquisa realizada no período do Projeto. Assim como os objetivos
específicos, que são considerados como sub-produtos do Objetivo
Geral. Além disso, deve-se apresentar os principais resultados
esperados do desenvolvimento desta Tese.

\chapter{Metodologia}
% (ENTRE 1 e 3 PÁGINAS)

Nesta Seção, apresenta-se a metodologia proposta para o
desenvolvimento da Tese. O proponente deve descrever as
atividades necessárias para a conclusão dos objetivos propostos. 

\chapter{Cronograma}

Esta Seção deve apresentar relação numerada de atividades (de estudo,
modelagem, especificação, implementação ou validação) que deverão ser
realizadas e o cronograma
destas atividades~\cite{vonNeumann:1966:TSR}.

\bibliography{bibliografia}
\bibliographystyle{abnt}

\chapter{Assinaturas}
\vspace{2cm}

\begin{center}
\rule{8cm}{.3mm}
\medskip

	Nome do Aluno\\
	Proponente

\end{center}

\vspace{4cm}

\begin{center}
\rule{8cm}{.3mm}
\medskip

	Nome do Professor\\
	Prof. Orientador

\end{center}
\end{document}

